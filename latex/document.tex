\documentclass[a4paper, titlepage, twoside, openright, appendixprefix, numbers=noendperiod]{scrreprt}

% % % % % % % % % % % % % % % % %
% WICHTIG!
% Das Dokument ist zweiseitig formatiert, es MUSS beidseitig gedruckt werden.
% Neue Kapitel beginnen immer rechts (Option openright)
% wird "numbers=noendperiod" gelöscht, reagiert das KOMASkript Dudenkonform: bei Überschriftennummerierung keine Endpunkte ohne Anhang; mit Endpunkten, wenn es einen Anhang gibt 
% % % % % % % % % % % % % % % % %

% Laden der Zusatzpakete (aus dem Ordner "input") 
% Paket für Input Encoding
% wenn utf8 nicht funktioniert, bitte ansinew (Windows) oder applemac (Mac) benutzen.
\usepackage[utf8]{inputenc}

% Font Encoding, u.A. für die korrekte Ausgabe in PDF-Dokumenten
\usepackage[T1]{fontenc}

% Anpassung der Sprache, in diesem Fall: Deutsch in neuer Rechtschreibung
\usepackage[english]{babel}

% Einstellung der Geometry des Layouts
\usepackage[textwidth=145mm,textheight=235mm,left=35mm,top=25mm,headsep=5mm]{geometry}

% optimierte Schrift für PDF-Dokumente
\usepackage{lmodern}

% Zum einfachen Einfügen von Grafiken
\usepackage{graphicx}

% Für lang Tabellen
\usepackage{longtable}

% Erweiternde Pakete für den Formelsatz
\usepackage{amsmath, amssymb, amsthm, amsfonts}

% Erstellt Verweise in PDF-Dokumenten. Die Verweise haben die Farbe schwarz, sind also nicht extra gekennzeichnet.
\usepackage[colorlinks,linkcolor=black,citecolor=black]{hyperref} 

% Paket für Aufzählungen. Zu verwenden wie itemize.
\usepackage{enumerate}

% Anführungszeichen
\usepackage[style=german]{csquotes} 

% "schöne" Tabellen
\usepackage{booktabs}

% Deckblatt für die Seminararbeit. Metadaten in titlepage.tex anpassen.
\usepackage{VOSTitle}

% Paket für die Selbstständigkeitserklärung, nutzt die Metadaten in titlepage.tex
\usepackage{VOSStatement}

% Für Zeilenabstände
\usepackage{setspace}

% Blindtext
\usepackage{lipsum}

% Paket für das Abkürzungsverzeichnis
\usepackage{acronym}

% Paket für die Zusammenfassung nach dem Titelblatt
\usepackage[style]{abstract}

% Paket für angepasste Bibliografie-Stile
\usepackage{bibgerm}

\usepackage{fancyhdr}
% Optional: Gleitobjekte nicht in andere Abschnitte fließen lassen 
%(Doku:http://mirror.informatik.uni-mannheim.de/pub/mirrors/tex-archive/macros/latex/contrib/placeins/placeins-doc.pdf) 
%\usepackage{placeins}

\usepackage[style=apa]{biblatex}
\usepackage{parskip}

% Laden weiterer Einstellungen
% einige andere Einstellungen oder Ergänzungen

% Festlegen des Titel-Stils der abstract-Umgebung
\renewcommand{\abstitlestyle}[1]{{\Large\bfseries\sffamily\noindent #1}\hfill}

% Änderung der Nummerierung der Formeln	
\numberwithin{equation}{section} 

% anderthalbfacher Zeilenabstand 
\onehalfspacing

% Globaler Seitenstil

\pagestyle{headings}
\fancyhf{} %
\pagestyle{fancy} %
\fancyfoot[LE,RO]{\thepage} %
\fancyhead[LO]{\nouppercase\rightmark} %
\fancyhead[RE]{\nouppercase\leftmark}%




% Parameter für das Titelblatt und die Selbstständigkeitserklärung
% % % % % %
% Bitte beachten Sie die Hinweise zum Ausfüllen.
% % % % % % 

% Lehrstuhl, an dem die Arbeit geschrieben wurde
\professur{Chair of Econometrics and Statistics}

% Art der wissenschaftlichen Arbeit
\thesistype{Research Seminar}

\fak{\enquote{Friedrich List} Faculty of Transport and Traffic Sciences}

% Namen und Matrikelnummern möglicher Autoren.

% % % % % % % % % % % % % % % % % % % % % % % % % %
%   Bitte die Autoren DER REIHE NACH auffüllen	  %
% % % % % % % % % % % % % % % % % % % % % % % % % %

% Bei nur einem Autor muss authorOne ausgefüllt werden
\authorOne{Henry Haustein}
\matrikelAuthorOne{4685025}

% Hat die Arbeit zwei Autoren, muss authorTwo ausgefüllt werden
\authorTwo{} 
\matrikelAuthorTwo{}

% Bei drei Gruppenmitgliedern ist auch authorThree zu belegen
\authorThree{} 
\matrikelAuthorThree{}

% NUR, FALLS TATSÄCHLICH BENÖTIGT, ANSONSTEN LEER LASSEN
% Für das vierte Gruppenmitglied
\authorFour{}
\matrikelAuthorFour{}

% Titel der Aufgabenstellung
\title{Developing a Reinforcement Learning Agent for the Swing-Up Cart-Pole Problem in a Real Environment}

% Betreuer am Lehrstuhls
\betreuer{Ankit Anil Chaudhari}

% Datum der Abgabe. \today ist der heutige Tag, bitte ggfs. ändern auf den 8. Januar oder wann immer Sie abgeben
\date{\today}


% % % % % % % % % % % % % % % %
% Beginn der document-Umgebung
% % % % % % % % % % % % % % % %
\begin{document}

% Erstellen des Titelblatts
\maketitle

\cleardoublepage

% Umstellung der Seitennummerierung auf römische Ziffern
\pagenumbering{roman}

% Zusammenfassung, wenn nicht benötigt, auskommentieren!
\selectlanguage{ngerman}

\begin{abstract}
\noindent
\lipsum[20-21]
\end{abstract}

% FALLS BENÖTIGT: Englische Zusammenfassung (in der Regel nicht in Seminararbeiten gefordert!)
%\selectlanguage{english}
%\begin{abstract}
%\noindent
%\lipsum[20-23]
%\end{abstract}
%\selectlanguage{ngerman}

\cleardoublepage

% Erstellen des Inhalts-, Abbildungs- und Tabellenverzeichnisses
\tableofcontents
\cleardoublepage
  
\phantomsection\addcontentsline{toc}{chapter}{Abbildungsverzeichnis} 
\listoffigures
\cleardoublepage

\phantomsection\addcontentsline{toc}{chapter}{Tabellenverzeichnis}
\listoftables
\cleardoublepage

% Einfügen des Abkürzungsverzeichnisses
% Bitte die Datei listofabbreviations.tex im Order um die eigenen Abkürzungen ergänzen.
\phantomsection\addcontentsline{toc}{chapter}{Abkürzungsverzeichnis}
% \chapter*{Abkürzungsverzeichnis}
% Der String XXXXXXXX hat keine wirkliche Bedeutung; die acronym-Umgebung verlangt einen Parameter, der angibt, wie breit die Abkürzungen in der Übersicht sein dürfen, in diesem Fall 8*X
\begin{acronym}[XXXXXXXX]
% \acro{BIP}{Bruttoinlandsprodukt}
% \acro{x}[\ensuremath{\bar{x}}]{Mittelwert}
\end{acronym}
\cleardoublepage

% Umstellung der Seitennummerierung auf wieder auf arabische Ziffern 
\pagenumbering{arabic}

% Hier beginnen die Kapitel und der eigentliche Inhalt.
% Es empfiehlt sich, den Inhalt einzelner Kapitel in separaten Dateien zu halten und diese hier mit \input{file} einzubinden.

%Einleitungskapitel, hier sind derzeit noch nachfolgende Kapitel drin, diese besser in eigene Dateien trennen
%Empfehlung: Je Kapitel eine tex-Datei anlegen und mit \input{...} in das Hauptdokument einbinden
%lipsum ist Blindtext und kann gelöscht werden!
\chapter{Einleitung}

%Kapitel mit wichtigen Beispielen
\chapter{Latex-Beispiele}


%Anhang
\appendix
\phantomsection\addcontentsline{toc}{chapter}{Anhang}
\chapter{Auswertung}

\section{Daten}
\section{Ergebnisse}

% Festlegen des Bibliografie-Stils, in diesem Fall die deutsche Variante des Standard-Stils plain
% Für Bachelor- und Masterarbeiten sollte biblatex oder natbib verwendet werden. Eine kurze Suche bei Google führt auf die Paketdokumentationen,
% aus der sich alles Weitere ergibt.
\bibliographystyle{gerplain}

% Einbinden der Bibliothek
\bibliography{beispielbib}
\thispagestyle{empty}

% Befehl zum Erstellen der Selbstständigkeitserklärung. Achtung: Es werden die Namen der Autoren, die in titlepage.tex festgelegt wurden, herangezogen und das Datum in \date{} verwendet.
% Befehl steht nur zur Verfügung, wenn das Paket VOSStatement eingebunden ist.
\cleardoublepage

\makestatement

% % % % % % % % % % % % % % % %
% Ende der document-Umgebung
% % % % % % % % % % % % % % % %
\end{document}