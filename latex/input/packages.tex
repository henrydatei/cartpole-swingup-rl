% Paket für Input Encoding
% wenn utf8 nicht funktioniert, bitte ansinew (Windows) oder applemac (Mac) benutzen.
\usepackage[utf8]{inputenc}

% Font Encoding, u.A. für die korrekte Ausgabe in PDF-Dokumenten
\usepackage[T1]{fontenc}

% Anpassung der Sprache, in diesem Fall: Deutsch in neuer Rechtschreibung
\usepackage[english]{babel}

% Einstellung der Geometry des Layouts
\usepackage[textwidth=145mm,textheight=235mm,left=35mm,top=25mm,headsep=5mm]{geometry}

% optimierte Schrift für PDF-Dokumente
\usepackage{lmodern}

% Zum einfachen Einfügen von Grafiken
\usepackage{graphicx}

% Für lang Tabellen
\usepackage{longtable}

% Erweiternde Pakete für den Formelsatz
\usepackage{amsmath, amssymb, amsthm, amsfonts}

% Erstellt Verweise in PDF-Dokumenten. Die Verweise haben die Farbe schwarz, sind also nicht extra gekennzeichnet.
\usepackage[colorlinks,linkcolor=black,citecolor=black]{hyperref} 

% Paket für Aufzählungen. Zu verwenden wie itemize.
\usepackage{enumerate}

% Anführungszeichen
\usepackage[style=german]{csquotes} 

% "schöne" Tabellen
\usepackage{booktabs}

% Deckblatt für die Seminararbeit. Metadaten in titlepage.tex anpassen.
\usepackage{VOSTitle}

% Paket für die Selbstständigkeitserklärung, nutzt die Metadaten in titlepage.tex
\usepackage{VOSStatement}

% Für Zeilenabstände
\usepackage{setspace}

% Blindtext
\usepackage{lipsum}

% Paket für das Abkürzungsverzeichnis
\usepackage{acronym}

% Paket für die Zusammenfassung nach dem Titelblatt
\usepackage[style]{abstract}

% Paket für angepasste Bibliografie-Stile
\usepackage{bibgerm}

\usepackage{fancyhdr}
% Optional: Gleitobjekte nicht in andere Abschnitte fließen lassen 
%(Doku:http://mirror.informatik.uni-mannheim.de/pub/mirrors/tex-archive/macros/latex/contrib/placeins/placeins-doc.pdf) 
%\usepackage{placeins}

\usepackage[style=apa]{biblatex}
\usepackage{parskip}