% % % % % %
% Bitte beachten Sie die Hinweise zum Ausfüllen.
% % % % % % 

% Lehrstuhl, an dem die Arbeit geschrieben wurde
\professur{Chair of Econometrics and Statistics}

% Art der wissenschaftlichen Arbeit
\thesistype{Research Seminar}

\fak{\enquote{Friedrich List} Faculty of Transport and Traffic Sciences}

% Namen und Matrikelnummern möglicher Autoren.

% % % % % % % % % % % % % % % % % % % % % % % % % %
%   Bitte die Autoren DER REIHE NACH auffüllen	  %
% % % % % % % % % % % % % % % % % % % % % % % % % %

% Bei nur einem Autor muss authorOne ausgefüllt werden
\authorOne{Henry Haustein}
\matrikelAuthorOne{4685025}

% Hat die Arbeit zwei Autoren, muss authorTwo ausgefüllt werden
\authorTwo{} 
\matrikelAuthorTwo{}

% Bei drei Gruppenmitgliedern ist auch authorThree zu belegen
\authorThree{} 
\matrikelAuthorThree{}

% NUR, FALLS TATSÄCHLICH BENÖTIGT, ANSONSTEN LEER LASSEN
% Für das vierte Gruppenmitglied
\authorFour{}
\matrikelAuthorFour{}

% Titel der Aufgabenstellung
\title{Developing a Reinforcement Learning Agent for the Swing-Up Cart-Pole Problem in a Real Environment}

% Betreuer am Lehrstuhls
\betreuer{Ankit Anil Chaudhari}

% Datum der Abgabe. \today ist der heutige Tag, bitte ggfs. ändern auf den 8. Januar oder wann immer Sie abgeben
\date{\today}
