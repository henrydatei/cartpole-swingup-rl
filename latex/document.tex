\documentclass[a4paper, titlepage, twoside, openright, appendixprefix, numbers=noendperiod]{scrreport}

% % % % % % % % % % % % % % % % %
% WICHTIG!
% Das Dokument ist zweiseitig formatiert, es MUSS beidseitig gedruckt werden.
% Neue Kapitel beginnen immer rechts (Option openright)
% wird "numbers=noendperiod" gelöscht, reagiert das KOMASkript Dudenkonform: bei Überschriftennummerierung keine Endpunkte ohne Anhang; mit Endpunkten, wenn es einen Anhang gibt 
% % % % % % % % % % % % % % % % %

% Laden der Zusatzpakete (aus dem Ordner "input") 
% Paket für Input Encoding
% wenn utf8 nicht funktioniert, bitte ansinew (Windows) oder applemac (Mac) benutzen.
\usepackage[utf8]{inputenc}

% Font Encoding, u.A. für die korrekte Ausgabe in PDF-Dokumenten
\usepackage[T1]{fontenc}

% Anpassung der Sprache, in diesem Fall: Deutsch in neuer Rechtschreibung
\usepackage[english]{babel}

% Einstellung der Geometry des Layouts
\usepackage[textwidth=145mm,textheight=235mm,left=35mm,top=25mm,headsep=5mm]{geometry}

% optimierte Schrift für PDF-Dokumente
\usepackage{lmodern}

% Zum einfachen Einfügen von Grafiken
\usepackage{graphicx}

% Für lang Tabellen
\usepackage{longtable}

% Erweiternde Pakete für den Formelsatz
\usepackage{amsmath, amssymb, amsthm, amsfonts}

% Erstellt Verweise in PDF-Dokumenten. Die Verweise haben die Farbe schwarz, sind also nicht extra gekennzeichnet.
\usepackage[colorlinks,linkcolor=black,citecolor=black]{hyperref} 

% Paket für Aufzählungen. Zu verwenden wie itemize.
\usepackage{enumerate}

% Anführungszeichen
\usepackage[style=german]{csquotes} 

% "schöne" Tabellen
\usepackage{booktabs}

% Deckblatt für die Seminararbeit. Metadaten in titlepage.tex anpassen.
\usepackage{VOSTitle}

% Paket für die Selbstständigkeitserklärung, nutzt die Metadaten in titlepage.tex
\usepackage{VOSStatement}

% Für Zeilenabstände
\usepackage{setspace}

% Blindtext
\usepackage{lipsum}

% Paket für das Abkürzungsverzeichnis
\usepackage{acronym}

% Paket für die Zusammenfassung nach dem Titelblatt
\usepackage[style]{abstract}

% Paket für angepasste Bibliografie-Stile
\usepackage{bibgerm}

\usepackage{fancyhdr}
% Optional: Gleitobjekte nicht in andere Abschnitte fließen lassen 
%(Doku:http://mirror.informatik.uni-mannheim.de/pub/mirrors/tex-archive/macros/latex/contrib/placeins/placeins-doc.pdf) 
%\usepackage{placeins}

\usepackage[style=apa]{biblatex}
\usepackage{parskip}

% Laden weiterer Einstellungen
% einige andere Einstellungen oder Ergänzungen

% Festlegen des Titel-Stils der abstract-Umgebung
\renewcommand{\abstitlestyle}[1]{{\Large\bfseries\sffamily\noindent #1}\hfill}

% Änderung der Nummerierung der Formeln	
\numberwithin{equation}{section} 

% anderthalbfacher Zeilenabstand 
\onehalfspacing

% Globaler Seitenstil

\pagestyle{headings}
\fancyhf{} %
\pagestyle{fancy} %
\fancyfoot[LE,RO]{\thepage} %
\fancyhead[LO]{\nouppercase\rightmark} %
\fancyhead[RE]{\nouppercase\leftmark}%




% Parameter für das Titelblatt und die Selbstständigkeitserklärung
% % % % % %
% Bitte beachten Sie die Hinweise zum Ausfüllen.
% % % % % % 

% Lehrstuhl, an dem die Arbeit geschrieben wurde
\professur{Chair of Econometrics and Statistics}

% Art der wissenschaftlichen Arbeit
\thesistype{Research Seminar}

\fak{\enquote{Friedrich List} Faculty of Transport and Traffic Sciences}

% Namen und Matrikelnummern möglicher Autoren.

% % % % % % % % % % % % % % % % % % % % % % % % % %
%   Bitte die Autoren DER REIHE NACH auffüllen	  %
% % % % % % % % % % % % % % % % % % % % % % % % % %

% Bei nur einem Autor muss authorOne ausgefüllt werden
\authorOne{Henry Haustein}
\matrikelAuthorOne{4685025}

% Hat die Arbeit zwei Autoren, muss authorTwo ausgefüllt werden
\authorTwo{} 
\matrikelAuthorTwo{}

% Bei drei Gruppenmitgliedern ist auch authorThree zu belegen
\authorThree{} 
\matrikelAuthorThree{}

% NUR, FALLS TATSÄCHLICH BENÖTIGT, ANSONSTEN LEER LASSEN
% Für das vierte Gruppenmitglied
\authorFour{}
\matrikelAuthorFour{}

% Titel der Aufgabenstellung
\title{Developing a Reinforcement Learning Agent for the Swing-Up Cart-Pole Problem in a Real Environment}

% Betreuer am Lehrstuhls
\betreuer{Ankit Anil Chaudhari}

% Datum der Abgabe. \today ist der heutige Tag, bitte ggfs. ändern auf den 8. Januar oder wann immer Sie abgeben
\date{\today}


% % % % % % % % % % % % % % % %
% Beginn der document-Umgebung
% % % % % % % % % % % % % % % %
\begin{document}

% Erstellen des Titelblatts
\maketitle

\cleardoublepage

% Umstellung der Seitennummerierung auf römische Ziffern
\pagenumbering{roman}

% Zusammenfassung, wenn nicht benötigt, auskommentieren!
% \selectlanguage{ngerman}

% \begin{abstract}
% \noindent
% \lipsum[20-21]
% \end{abstract}

% FALLS BENÖTIGT: Englische Zusammenfassung (in der Regel nicht in Seminararbeiten gefordert!)
%\selectlanguage{english}
%\begin{abstract}
%\noindent
%\lipsum[20-23]
%\end{abstract}
%\selectlanguage{ngerman}

% \cleardoublepage

% Erstellen des Inhalts-, Abbildungs- und Tabellenverzeichnisses
\tableofcontents
\cleardoublepage
  
\phantomsection\addcontentsline{toc}{chapter}{Abbildungsverzeichnis} 
\listoffigures
\cleardoublepage

\phantomsection\addcontentsline{toc}{chapter}{Tabellenverzeichnis}
\listoftables
\cleardoublepage

% Einfügen des Abkürzungsverzeichnisses
% Bitte die Datei listofabbreviations.tex im Order um die eigenen Abkürzungen ergänzen.
\phantomsection\addcontentsline{toc}{chapter}{Abkürzungsverzeichnis}
% \chapter*{Abkürzungsverzeichnis}
% Der String XXXXXXXX hat keine wirkliche Bedeutung; die acronym-Umgebung verlangt einen Parameter, der angibt, wie breit die Abkürzungen in der Übersicht sein dürfen, in diesem Fall 8*X
\begin{acronym}[XXXXXXXX]
% \acro{BIP}{Bruttoinlandsprodukt}
% \acro{x}[\ensuremath{\bar{x}}]{Mittelwert}
\end{acronym}
\cleardoublepage

% Umstellung der Seitennummerierung auf wieder auf arabische Ziffern 
\pagenumbering{arabic}

% Hier beginnen die Kapitel und der eigentliche Inhalt.
% Es empfiehlt sich, den Inhalt einzelner Kapitel in separaten Dateien zu halten und diese hier mit \input{file} einzubinden.

%Einleitungskapitel, hier sind derzeit noch nachfolgende Kapitel drin, diese besser in eigene Dateien trennen
\chapter{Introduction}

- Artificial neural networks sind in den letzten Jahren immer wichtiger geworden, viele populäre Anwendungen wie Spracherkennung (Deng et al. 2013), Bilderkennung (Rowley et al. 1998) und maschinelles Übersetzen (Zhang & Zong, 2015; Bengio, 2012) basieren auf ihnen
- Artifical neural networks sind ein Teilgebiet des maschinellen Lernens, ein anderes Teilgebiet, welches weit weniger populär ist, ist das Reinforcement Learning (Jordan, 2015)
- dabei sind auch im reinforcement learning beeindruckende Fortschritte erzielt worden, wie z.B. AlphaGo, das 2016 den Weltmeister im Go besiegte (Google DeepMind, 2020; Silver et al., 2016; Silver et al., 2017)
- der einstieg in reinforcement learning ist dank zahlreicher Videos und Tutorials im Internet relativ einfach, zudem gibt es viele Bibliotheken (wie z.B. OpenAI Gym), die die Implementierung von reinforcement learning Algorithmen erleichtern (Nicholas Renotte, 2021; Stable Baselines 3, 2024; sentdex, 2019)
- Das Buch "Reinforcement Learning: An Introduction" von Sutton und Barto ist ein Standardwerk in diesem Bereich und bietet eine umfassende Einführung in die Thematik und ist zudem unter creative commons lizensiert, sodass es kostenlos verfügbar ist (Sutton & Barto, 2018)

- Eines von den standard Beispielen in der Literatur ist das Cart-Pole-Problem, bei dem ein Wagen einen Stab balancieren muss, der auf ihm befestigt ist. dieses problem wird auch in vielen Videos und Tutorials im Internet behandelt (Sutton & Barto, 2018)
- dieses problem ist auch in OpenAI Gym implementiert, sodass es einfach ist, es zu verwenden und zu experimentieren und so in wenigen Zeilen Code ein reinforcement learning agent zu trainieren, der das Cart-Pole-Problem löst. Die Verwendung der Bibliothek Stable Baselines 3 macht es noch einfacher, da sie viele vorgefertigte reinforcement learning Algorithmen enthält, die einfach zu verwenden sind, so unter anderem auch Proximal Policy Optimization (PPO), das in vielen reinforcement learning Anwendungen erfolgreich eingesetzt wird (Arxiv Insights, 2018)
- das von OpenAI implementierte Environment ist aber nur simuliert, sodass der Ausgangszustand, das Pendel in der oberen Position, leicht wieder herzustellen ist. Wenn man das Cart-Pole-Problem in der Realität lösen möchte, so muss man manuell das Pendel nach jedem erfolglosen Versuch wieder in die obere Position bringen, was das Training deutlich verlängert
- wenn man also den trainingsprozess beschleunigen möchte, so muss einen reinforcement learning agent entwickeln, der das Pendel von seiner unteren Position in die obere Position bringt, um so das Cart-Pole-Swing-Up-Problem oder inverted-pendulum-problem in der Realität zu lösen
- in der Literatur gibt es nur wenige Arbeiten, die das Problem in der Realität lösen, und die meisten davon verwenden einfache reward Funktionen und simulieren das Environment, sodass es keine realen Herausforderungen gibt, die das Problem in der Realität lösen
- diese Arbeit beschäftigt sich mit der Entwicklung eines reinforcement learning agents, der das Swing-Up-Cart-Pole-Problem in der Realität löst, und dabei ein realistisches Environment verwendet, das auf einem Raspberry Pi basiert und ein echtes Cart und Pendel verwendet. Es beschreibt die Herausforderungen, die bei der Entwicklung eines solchen Systems auftreten, und die Lösungen, die gefunden wurden, um diese Herausforderungen zu bewältigen, sowie zukünftige Forschungsansätze, um das System weiter zu verbessern
\chapter{Background and Literature Review}

\section{Fundamentals of Reinforcement Learning}
- Reinforcement learning (RL) is a type of machine learning where an agent learns to make decisions by taking actions in an environment to maximize cumulative reward. This approach is inspired by behaviorist psychology, where learning is achieved through interactions with the environment and is reinforced by rewards and punishments. RL has gained significant attention due to its successful applications in various domains such as robotics, game playing, and autonomous systems. (Sutton & Barto, 1998)

- Fundamental Concepts of Reinforcement Learning
- Agent and Environment: In RL, the learning process involves an agent and an environment. The agent interacts with the environment by performing actions, and the environment responds by providing feedback in the form of rewards or penalties. The objective of the agent is to learn a policy that maximizes the total reward over time (Sutton & Barto, 1998).
- State (S): Represents the current situation or configuration of the environment.
- Action (A): The set of all possible moves the agent can make.
- Reward (R): The feedback from the environment based on the action taken by the agent.
- The agent's goal is to find a policy, π, that maps states to actions in a way that maximizes the expected cumulative reward, known as the return (Sutton & Barto, 1998).
- Policy (π): A policy is a strategy used by the agent to decide the next action based on the current state.
- Value Function (V): The value function estimates the expected return (total reward) starting from a state and following a particular policy. It helps in evaluating the goodness of states (Szepesvári, 2010).
- Model-Free RL: The agent learns a policy or value function without understanding the underlying model of the environment. Examples include Q-Learning (Watkins & Dayan, 1992) and PPO (Pan et al 2018).
- Model-Based RL: The agent learns a model of the environment (transition probabilities and reward function) and uses it to plan actions (Kaelbling, Littman, & Moore, 1996).

- Key Algorithms in Reinforcement Learning
- Dynamic Programming: Dynamic programming (DP) methods require a complete model of the environment. They are used to compute optimal policies by iteratively improving value functions. Examples include Policy Iteration and Value Iteration.
- Monte Carlo (MC) methods learn directly from episodes of experience. They do not require knowledge of the environment's model and estimate value functions based on sample returns .
- Temporal-Difference (TD) learning combines ideas from DP and MC methods. It updates value estimates based on the difference between consecutive estimates. (Sutton & Barto, 1998)

- Exploration vs. Exploitation: One of the central challenges in RL is the trade-off between exploration (trying new actions to discover their effects) and exploitation (choosing actions that are known to yield high rewards). Various strategies such as ε-greedy, softmax, and Upper Confidence Bound (UCB) are employed to balance this trade-off (Auer, Cesa-Bianchi, & Fischer, 2002).

- Deep Reinforcement Learning: With the advent of deep learning, RL has evolved into deep reinforcement learning (DRL), where neural networks are used to approximate value functions or policies. Notable advancements include: Deep Q-Network (DQN): Combines Q-Learning with deep neural networks to handle high-dimensional state spaces, such as those in Atari games (Mnih et al., 2015). Actor-Critic Methods: Use two neural networks, one for the policy (actor) and one for the value function (critic). Examples include A3C (Asynchronous Advantage Actor-Critic) and PPO (Proximal Policy Optimization) (Mnih et al., 2016; Schulman et al., 2017).

(hier kommt auch eine kurze Erklärung zum PPO Algorithmus hin)

\section{The Cart-Pole Problem and Swing-Up Dynamics}

\section{Review of Related Work in Real-World Environments}
\chapter{System Description}

- System war schon durch vorheige Experimente weitgehend aufgebaut und besteht aus 3D gedruckten Teilen, einem Stepper Motor, Kamera, 2 metallischen Stangen, Zahnriemen und diversen mechanischen Teilen (siehe Foto)
- 2 Fotos: von "hinten" und "vorne"
- Anpassungen wurden im Verlauf der Experimente vorgenommen
- einige der beweglichen Teile haben sich im Laufe der Experimente verschoben, sodass Klammern angebracht werden mussten, um die Teile zu fixieren

\section{Hardware Components}
\subsection{Cart}
- Detailfoto vom Cart
- auf einem Holzbrett im Abstand von 95 cm 2 Halterungen montiert für Metallstangen, auf denen das Cart fahren kann
- auf einer Halterung ist eine Umlenkrolle in Form eines Zahnrades montiert, die den Zahnriemen führt, auf der anderen Halterung ist der Stepper Motor montiert
- der Zahnriemen muss sehr straff gespannt sein, sonst rutscht er durch und das Cart bewegt sich nicht
- mehrere Halterungen für ein Gewinde, an dem das Pendel hängt
- Pendel ist 3D gedruckt und besteht aus 5 Teilen, die zusammengesteckt werden: 1 blauer breiter Stab und 4 Teile um den Stab, wovon 2 für die Verbindung von Pendel und Gewinde sorgen und 2 für zusätzliches Gewicht am Ende des Pendels sorgen
- ein Stück Plastik mit gelbem Klebeband ist auch am Pendel befestigt, um es der Kamera zu ermöglichen, das Pendel in jeder Position zu erkennen
- insgesamt ist das Pendel 35 cm lang und wiegt 128 g
- eigentlich verlangt das Cart-Pole-Experiment eine reibungslose Oberfläche, auf der das Cart fahren kann, dieses wurde versucht durch Rollen, die um die Metallstangen angebracht sind, zu erreichen, jedoch war die Oberfläche nicht reibungslos genug, sodass das Cart nicht reibungslos fahren konnte. Es ist leider verhältnismäßig viel Kraft notwendig, um das Cart zu bewegen, was die Experimente erschwert hat. Aber laut Escobar et al. (2020) ist das kein Problem, da das Cart-Pole-Experiment auch mit Reibung gelöst werden kann.

\subsection{Stepper Motor}
- Detailfoto vom Stepper Motor
- Stepper Motor dreht ein Zahnrand, auf dem sich der Zahnriemen befindet, dadurch wird das Cart bewegt
- bei verschiedenen Experimenten wurden unterschiedliche Stromsträken für den Stepper Motor getestet, der dadurch heiß wurde, deswegen wurde ein Lüfter montiert, der Lüfter ist vom selben Typ wie der Lüfter der einen Raspberry Pi kühlt und wird mit 3.3V betrieben
- verschiedene Stepper Motoren wurden im Verlauf der Experimente probiert, alle vom Typ NEMA 17, 1 Motor mit der Bezeichnung MOT-AN-S-060-005-042-L-A-AAAA von igus, 2 Motoren mit der Bezeichnung 17HS19-2004S1 von stepperonline, wovon einer nicht akkurat war
- der Motor von igus verträgt eine Stromstärke von 1.8A, die anderen Motoren vertragen 2A
- die Stromstärke wurde im Verlauf der Experimente verändert, um die beste Leistung zu erzielen, ohne den Motor zu überhitzen

\subsection{Stepper Motor Control}
- alle Motoren wurden mit einem Makeblock 2H Microstep Driver angesteuert, der mit einem Arduino verbunden ist (genauere Bezeichnung des Arduinos?)
- der Arduino ist mit einem Raspberry Pi und Microusb verbunden
- alle Stepper Motoren wurden mit einer Schrittwinkelauflösung von 1.8° betrieben (12800 Schritte pro Umdrehung)

\subsection{Camera}
- Detailfoto von der Kamera
- Kamera ist eine Picam, die in einem 3D gedruckten Gehäuse montiert ist, das an dem Cart befestigt ist
- Kamera ist mit einem Raspberry Pi verbunden, der die Bilder der Kamera verarbeitet
- Position der Kamera darf sich nicht verändern, da sonst die Winkel des Pendels, die erkannt werden sollen, nicht mehr korrekt sind
- Deaktivieren des Autofocus der Kamera, um die Bilder schneller zu machen und immer die gleiche Schärfe zu haben
- Einsatz von Klebeband, um die Kamera zu fixieren

\section{Software Architecture}
\subsection{Reinforcement Learning Agent}
- Entwicklung einer eigenen Environment auf Basis von OpenAI Gym, die die Cart-Pole-Umgebung nachbildet
- wichtigste Änderung: Überschreiben der step-Funktion, damit die ausgewählten Aktionen in Nachrichten umgewandelt werden, die an den Arduino gesendet werden, um das Cart zu bewegen
- Algorithmus: PPO, Implementierung von stable baselines 3
- Verarbeitung der Bilder der Kamera in einem eigenen Skript, welches unabhängig vom Reinforcement Learning Agent läuft, um die Winkel des Pendels zu bestimmen
- Winkel starten bei 0 in der oberen Position und gehen bis 180 Grad in der unteren Position, auf der linken Seite des Pendels von 0 bis 180 Grad, auf der rechten Seite von 0 bis -180 Grad
- Senden der Winkel und weiterer Informationen über ZeroMQs Implementeriung eines TCP Sockets an den Reinforcement Learning Agent
- Vorteil: bessere Nutzung der 4 Kerne des Raspberry Pis (1 Kern für RL Agent, 1 Kern für Bildverarbeitung), Bilder können häufiger gemacht werden, akuratere Bestimmung der Winkelgeschwidigkeit durch mehr Messungen, Reduktion des Delays in den Entscheidungen des RL Agents, da die Bilder nicht mehr verarbeitet werden müssen, bevor der RL Agent eine Aktion auswählt

\subsection{Control Algorithms}
- Kommunikation des Raspberry Pi mit dem Arduino über USB als serielle Schnittstellenverbindung
- eigenes Protokoll, Nachrichten werden in Form von Strings übertragen, Strings haben die Form "<letter,number>", die Buchstaben stehen für die Art der Nachricht, die Zahlen für die Werte
- folgende Arten von Nachrichten sind möglich: (Tabelle benutzen) "v" um die Geschwindigkeit im Steps/Sekunde festzulegen, "a" für die Beschleunigung in Steps/Sekunde^2, "m" um den Stepper Motor zu einer bestimmten Position - ausgehend von einem Home-Punkt 0, der beim Start des Ardoino gesetzt wird - zu bewegen, "r" für eine Bewegung relativ zur eigenen Position, "h" immer mit dem Wert 0, um die aktuelle Position als Home-Punkt zu setzen, "s" um die Schritte pro Umdrehung zu setzen und "p" immer mit dem Wert 0, um die aktuelle Position des Stepper Motors abzufragen
- Arduino kennt immer die aktuelle Position durch mitzählen der durchgeführten Schritte und stoppt den Motor automatisch, wenn das Cart das Ende der Stangen erreicht. Dafür ist es notwendig, den Home-Punkt immer in der Mitte des Streckenabschnitts zu setzen, den das Cart zurücklegen soll, um sicherzustellen, dass das Cart immer anhält, wenn es das Ende der Stange erreicht, die maximale Anzahl der Umdrehungen des Stepper Motors bevor das Cart das Ende der Stangen erreicht, wurde mit 9 ermittelt.
- Der Raspberry Pi sendet die Nachrichten an den Arduino, um das Cart zu bewegen, und durch die Asynchronität des Systems kann der Raspberry Pi in der Zwischenzeit andere Berechnungen durchführen, um die nächste Aktion des Reinforcement Learning Agents zu bestimmen, während das Cart sich bewegt
- Der Ardoino führt (bei Angabe einer Geschwindigkeit) die Bewegung des Carts solange aus, bis er eine neue Nachricht bekommt oder das Cart das Ende der Stange erreicht hat

- weiterhin wurde eine funktion eingebaut, die ermittelt, ob das Büro leer ist, um diese Mitarbeiter nicht durch die Geräusche zu stören, die das Cart macht, wenn es sich bewegt
- ob das Büro leer ist, hängt aktuell nur von der Zeit ab, aktuell wird davon ausgegangen, dass das Büro von 8:30 am Morgen bis 18:30 am Abend von Montag bis Freitag besetzt ist. Außerhalb dieser Zeiten wird davon ausgegangen, dass das Büro leer ist und das Cart sich bewegen kann ohne jemanden zu stören.
- Die Überprüfung, ob das Büro leer ist, wird bei jedem reset des RL Agents durchgeführt (Aufruf der reset-Funktion der Gym Environment). Ist das Büro nicht leer, so wird 10 Minuten gewartet, bis erneut geprüft wird.

\chapter{Experimental Setup}

- auf der Suche nach einem System, was das Inverse Cartpole Problem löst, wurde iterativ vorgegangen, das heißt es wurde ein Experiment durchgeführt, die Ergebnisse wurden analysiert und das Experiment wurde angepasst, um die Ergebnisse zu verbessern
- Fokus der ersten Experimente lag darauf, die Winkelbestimmung mittels Kamera zu verbessern
- Fokus der späteren Experimente lag darauf, die übrige Hardware, insbesondere den Stepper Motor, zu verbessern
- Fokus der letzten Experimente lag darauf, die Software zu verbessern, insbesondere die Reward-Funktion des Reinforcement Learning Agents

\section{Determining the angle}
- in Experimenten vor mir wurde der Winkel des Pendels durch die Erkennung von farbigen Rechtecken auf dem Pendel und auf dem Stück Plastik, welches so am Pendel befestigt ist, dass es auf der anderen Seite des Rotationspunktes des Pendels ist, bestimmt
- Die Farben waren Gelb und Violett und mithilfe der Mitte der Bounding Box um die erkannten Farben konnte dann der Winkel bestimmt werden, durch eine Verbindung des Mittelpunktes und des Rotationspunktes des Pendels, welcher auch im Bild erkennbar war, aber durch hartcodierte Pixelkoordinaten festgelegt war. Die Kamera durfte sich also nicht bewegen, da sonst wie Winkelbestimmung nicht mehr korrekt war.

- weitere Experimente: Anbringen von Symbolen, hier Kreis und Dreieck, auf dem Pendel und auf dem Stück Plastik, sodass die Farberkennung nicht mehr notwendig war, sondern die Symbole erkannt werden konnten.
- Ansatz: Erkennung der verschiedenen geometrischen Figuren anhand ihrer Konturen und Approximation mittels des Douglas-Peucker Algorithmus, Implementation in OpenCV's approxPolyDP(). Wahl von Dreieck und Kreis aufgrund ihrer unterschiedlichen Anzahl von Ecken und sehr unterschiedlichen Konturen.
- Filterung über Momente einer Kontur mittels OpenCV's moments() Funktion, z.B. um nur Formen einer bestimmten Größe zu erkennen
- Erkennung der Kreise auf anderem Weg: Erkennung des Kreises mittels OpenCV's Hough Circle Transform, die zuerst mögliche Mittelpunkte und dann passende Radien bestimmt, um Kreise zu erkennen. 

- nächste Experimente versuchen den Weg der Farberkennung zu verbessern
- Transformation des Bildes vom RGB-Farbraum in den HSV-Farbraum, der weniger empfindlich für Beleuchtungsänderungen ist
- Suchen von komplementären Farben im HSV Farbraum liefert zum Beispiel die Farben blau und gelb
- Zusätzlich wurde Sonneneinstrahlung von außen durch Herunterlassen des Sonnenschutzes versucht zu verringern und dauerhafte Beleuchtung des Zimmers mittels künstlichem Licht 
- Da das Pendel bereits in blauem Plastik gedruckt wurde, wird nur auf dem Stück Plastik ein gelbes Stück Klebeband angebracht
- das Pendel ist nicht immer komplett zu sehen, weswegen der Einsatz einer Bounding-Box und dann nehmen des Mittelpunktes zu falschen Winkeln führen würde
- stattdessen Bestimmen des Moments m00 von OpenCV's moments() Funktion, welcher die Koordinates des Schwerpunktes angibt. Dieses Vorgehen verhindert auch Probleme mit der Winkelbestimmung, falls Konturen nicht glatte Kanten erkannt werden, z.B. wenn Schatten ungünstig geworfen werden; der Schwerpunkt bleibt relativ an der selben Stelle

\section{Optimizing Hardware}
- Bei Experimenten wurde klar, dass das Setup in der Konfiguration, in der es sich bei vorgegangen Experimenten befand, nicht optimal war und nicht in der Lage war, das Pendel nach oben zu bekommen
- Verschiedene Möglichkeiten der Optimierung wurden identifiziert: das Pendel leichter machen, sodass weniger Kraft nötig ist, um es nach oben zu bekommen, den Strom des Stepper Motors erhöhen, um mehr Kraft zu erzeugen, oder die Geschwindigkeit mit der der Stepper Motor das Cart bewegt, erhöhen
- Viele dieser Experimente dauerten nur wenige Minuten, da bereits abzusehen war, dass sie nicht erfolgreich sein würden und das Pendel nicht nach oben gebracht werden würde

- Das Pendel wurde leichter gemacht, indem das Gewicht am Ende des Pendels entfernt wurde, sodass das Pendel nur noch aus dem blauen Stab und den 2 Verbindungsstücken besteht. Das Gewicht des Pendels wurde dadurch von XX g auf YY g reduziert % TODO: Gewicht des Pendels bestimmen

- In seiner urprünglichen Kofiguration wurde der Stepper Motor vom Stepper Motor Driver mit einem Strom von 0.61 A (RMS, Root Mean Square) betrieben, was einen Peakstrom von 0.86 A bedeutet
- Testen verschiedener Stepper Motoren: 1x MOT-AN-S-060-005-042-L-A-AAAA von igus und 2x 17HS19-2004S1 von stepperonline bei verschiedenen Strömen
- Der verbautete Motor MOT-AN-S-060-005-042-L-A-AAAA von igus verträgt eine Stromstärke von 1.8 A, der Stepper Motor Driver kann maximal einen RMS Strom von 2 A liefern (Peak 2.83 A), der Stepper Motor 17HS19-2004S1 von stepperonline verträgt eine Stromstärke von 2 A 

- Experimente mit verschiedenen Geschwindigkeiten und Beschleunigungen des Stepper Motors, um die beste Leistung zu erzielen. Das Cart wurde immer mit maximaler Geschwindigkeit und zufälligen Actions (links oder rechts) getestet, um zu schauen, ob das Pendel nach oben gebracht werden kann. Geteste Geschwindigkeiten 50000 Steps/Sekunde - 70000 Steps/Sekunde in 5000er Schritten, geteste Beschleunigungen 1000000 Steps/Sekunde^2 - 10000000 Steps/Sekunde^2 in 1000000er Schritten

\section{Optimizing Software}
- Zur Software gehören 2 Teile: zum einen der Umgang mit den bestimmten Winkeln des Pendels, zum anderen der RL Agent und dort insbesondere die Reward-Funktion

- welche Informationen aus den Kamerabildern genutzt werden, hing immer größtenteils von der Reward-Funktion ab, aber grundlegend wurde immer der Winkel und die Winkelgeschwidigkeit ermittelt.
- verschiedene Methoden zur Winkelbestimmung wurden in der Section Determining the angle beschrieben, für die Winkelgeschwindigkeit wurde die Differenz der Winkel in den letzten 2 Frames genommen und deren Zeitdifferenz genutzt, um die Winkelgeschwindigkeit zu bestimmen
- es wurde auch versucht die Winkelgeschwidigkeit über eine FIFO-Queue zu bestimmen, die entweder die letzten 3 oder 5 Winkel mit Zeitstempel enthält, um so die Winkelgeschwidigkeit über einen größeren Zeitraum zu bestimmen und so zu Glätten
- alle relevanten Informationen (z.B. aktuelle Zeit, 1-5 letzten Winkel, Winkelgeschwidigkeit) wurden dann über ZeroMQ in eine Queue der Länge 1 gelegt, die der RL Agent abfragen konnte, um die aktuellste Beobachtung zu erhalten
- es kann passieren, dass der RL Agent schneller Infors aus der ZeroMQ-Queue abruft, als neue Beobachtungen erzeugt werden können und der RL Agent in diesem Fall eine Nachricht mit nur Nullen enthält. Um zu unterscheiden, ob die Nachricht nur Nullen enthält weil der Pole gerade oben ist und sich nicht bewegt oder weil der RL Agent zu schnell ist, wurde ein Flag eingeführt, welches 1 ist, wenn der Pole oben ist. Eine Nachricht, die nur aus Nullen besteht, kann dann nie gültig sein. In diesem Fall hat der RL Agent die letzte Nachricht gespeichert und nutzt diese als aktuellste Beobachtung.
- Observation space enthält die 1-5 Winkel, die Winkelgeschwindigkeit, die aktuelle Position und Geschwindigkeit des Carts und ob das Pendel oben oder unten ist. Das Pendel ist dann oben, wenn der Betrag des Winkels kleiner als 12 Grad ist, wie von Nagendra (2017) vorgeschlagen. Ansonten ist das Pendel unten.

- verschiedene action spaces wurden mit der simplen Reward-Funktion getestet
- in Anlehnung an Cartpole Umgebung von OpenAI Gym wurde ein action space von 2 Aktionen getestet, die das Cart entweder nach links oder nach rechts bewegen
- Erweiterung auf mehr Aktionen, die das Cart in 10$^\circ$ Abstufungen mit einer Aktion von ganz links nach ganz rechts bewegen können (insgesamt 648 Aktionen möglich)
- Aktionen sind Geschwindigkeiten, die das Cart in Steps/Sekunde bewegen, wobei die Geschwindigkeit und Vorzeichen in 1000er Schritten von 0 bis 60000 Steps/Sekunde gewählt werden kann
- Reduktion auf 2000er Schritte und 5000er Schritte, um die Anzahl der Aktionen zu reduzieren, da die Anzahl der Aktionen die Trainingszeit erhöht
- Reduktion auf 2 Aktionen, Geschwindigkeit von -60000 und 60000, aber solange der Pole unten ist, wird jede Aktion für mindestens 0.1 Sekunden ausgeführt, um Winkelgeschwindigkeit aufzubauen. Die 0.1 Sekunden wurden per time.sleep(0.1) realisiert. Wenn der Pole oben ist, entfällt die Zeitverzögerung, der Pole kann durch viele Bewegungen balanciert werden.

- verschiedene Reward-Funktionen wurden getestet, um den RL Agent zu trainieren, das Pendel nach oben zu bekommen
- simple Reward-Funktion, in der der Reward $r_{simple}=\cos(\theta)$ ist, wobei $\theta$ der Winkel des Pendels ist. Durch die Defition der Winkel ist in der oberen Position der Winkel 0, welches einen Reward von 1 ergibt, in der unteren Position $\pm$ 180 Grad, was einen Reward von -1 ergibt. Diese Reward Funktion findet sich z.B. bei Doya (2000) oder bei Wawrzynski & Pacut (2004)
- H. Kimura & S. Kobayashi (1999) haben folgende Reward-Funktion vorgeschlagen: \begin{align}
    r_{Kimura,Kobayashi} = \begin{cases}
        -1 & \vert\theta\vert \ge 0.8\pi \\
        -3 & \vert\dot{\theta}\vert \ge 10 \\
        1 & \vert\theta\vert < 0.133\pi \land \vert\dot{\theta}\vert < 2 \\ 
        0 & \text{sonst}
    \end{cases}
\end{align} hier mit $\theta$ als Winkel in Radiant und $\dot{\theta}$ als Winkelgeschwindigkeit in Radiant pro Sekunde.
- Escobar et al. (2020) haben folgende Reward-Funktion vorgeschlagen, die leicht modifiziert wurde, um sie an das Setup anzupassen. Im originalen Paper fließt die verwendete Kraft ein, um den RL Agent zu motivieren, mit möglichst wenig Kraft das Ziel zu erreichen. Ein zweiter Teil der Reward-Funktion liefert einen negativen Reward, wenn das Cart sich außerhalb des zulässigen Bereiches befindet. Das ist bei diesem Setup nicht möglich, weswegen dieser Teil der Reward-Funktion weggelassen wurde. \begin{align}
    r_{Escobar} = -0.01\left(0.01\cdot\vert x\vert^2 + 0.1\cdot\vert\theta\vert^2 + 5\cdot\vert \dot{x}\vert^2\right)
\end{align} wobei $x$ die Position und $\dot{x}$ die Geschwindigkeit des Carts ist. 
- Aufgrund von hohen Winkelgeschwindigkeiten, die zwar nötig sind, um den Pole nach oben zu bekommen, aber in der Balancierungsphase, wenn der Pole oben ist, hinderlich sind, wurde noch eine Reward-Funktionen getestet, die hohe Winkelgeschwindigkeiten nur bis einem bestimmten Punkt fördern. \begin{align}
    r_{highVelocity} = \begin{cases}
        \cos(\theta) & \vert\theta\vert \le 12^\circ \\
        \left|\frac{\dot{\theta}}{100}\right| & \vert\theta\vert > 12^\circ
    \end{cases}
\end{align}
- Aus vorgegangen Experimenten hatte sich auch eine Reward-Funktion ergeben, diese nutzt die letzten 5 Winkel und bestimmt eine Winkeländerung. Zugleich besteht diese Reward-Funktion aus mehreren Teilen, die den Winkel belohnen, zu hohe Winkelgeschwindigkeiten bestrafen, den RL Agent bestrafen, wenn sich das Cart zu weit aus der mittleren Position bewegt und den RL Agent bestraft, wenn der Winkel mit fortgeschrittenem Training unten bleibt. \begin{align}
    angle_reward &= \exp\left(\frac{\cos(\theta_1) + \cos(\theta_2) + \cos(\theta_3) + \cos(\theta_4) + \cos(\theta_5)}{\alpha}\right) \\
    angular_velocity_penalty &= \begin{cases}
        angle_reward\cdot\frac{\Delta\theta_1 + \Delta\theta_2 + \Delta\theta_3 + \Delta\theta_4}{4\cdot 2\pi} & \vert\theta_5\vert \le 12^\circ \\
        0 & \text{sonst}
    \end{cases} \\
    no_swing_up_penalty &= \begin{cases}
        \frac{time steps since training start}{50000} & \vert\theta_5\vert \ge 168^\circ \\
        0 & \text{sonst}
    \end{cases} \\
    r_{complex} &= angle_reward - \beta\cdot position_penalty - \gamma\cdot angular_velocity_penalty - \delta\cdot no_swing_up_penalty
\end{align}
wobei $\theta_5$ der aktuellste Winkel ist und $\Delta\theta_i$ für $i=1,...,4$ die absolute Änderung von $\theta_i$ zu $\theta_{i+1}$ ist. $\alpha$ bis $\delta$ erlauben eine unterschiedliche Gewichtung der einzelnen Teile der Reward-Funktion.
\chapter{Results and Discussion}

\section{Results of the Experiments}

\subsection{First Set of Experiments: Determining the angle}
- Bei diesen Experimenten wurde noch kein RL Agent eingesetzt, sondern die erkennten Winkel direkt auf dem Bildschirm ausgegeben, um die Winkelbestimmung zu überprüfen.
- Problem bei Erkennung der farbigen Rechtecke auf Pendel und Stück Plastik,  dass von außen scheinendes Sonnenlicht die Farben veränderte, sodass die Farberkennung nicht mehr funktionierte. insbesondere bei dem Violett war dieser Effekt besonders stark ausgeprägt.
- Lösung durch anbringen von Symbolen, die verändern ihre Farbe nicht
- Trotz viel Kalibierung und Ausprobieren von verschiedenen Varianten der Formen (ausgedruckt vs. selbst gezeichnet) war keine zuverlässige Erkennung möglich, es wurden zu viele Elemente im Bild, wie z.B. Schrauben oder Objekte im Hintergrund (Personen, die durch das Bild gehen, Stühle, etc) erkannt, häufig als Kreis
- auch Filterung über OpenCVs moment() funktion liefert keine zuverlässige Erkennung
- Erkennung der Kreise über Hough Circle Transformation: Die Bestimmung der Dreiecke ist recht zuverlässig, aber die Bestimmung der Kreise, selbst mit dieser Methode ist stark fehleranfällig, oft werden Kreise nicht erkannt. Da diese Funktion aber in verschiedenen Quellen immer gute Ergebnisse liefert, vermute ich, dass das Bild der Kamera zu viel Rauschen enthält.
- verbesserter Ansatz zur Farberkennung liefert zuverlässige Winkel

\subsection{Second Set of Experiments: Optimizing Hardware}
- Pendel leichter machen hat nichts gebracht: Wenn das Pendel leichter ist, hat die Trägheit des Pendels aber nicht ausgereicht, um das Pendel nach oben zu bekommen

- bei hohen Strömen (1.51 A RMS, 2.14 A Peak und 2 A RMS, 2.83 A Peak) wird der Stepper Motor sehr heiß, weswegen ein Lüfter montiert wurde, der den Motor kühlt
- einer der zwei Stepper Motoren von stepperonline war nicht akkurat, die berechnete Position des Arduino hat nicht mit der tatsächlichen Position des Carts übereingestimmt
- geringe Ströme bei diesem Motor führen zu keiner Verbesserung des Motors bezüglich der Genauigkeit
- der andere Motor von stepperonline war akkurat, auch bei höhen Strömen
- Experimente mit verschiedenen Strömen lieferten einen RMS Strom von 1.2 A (Peak 1.69 A) als guten Tradeoff zwischen Kraft des Motors und Überhitzung des Motors für den Stepper Motor von stepperonline
- der Lüfter kann dann mit 3.3V betrieben werden und ist damit relativ leise

- Erkenntnisse aus Experimenten mit maximaler Geschwindigkeit: Je höher die Geschwindigkeit des Stepper Motors gewählt wird, desto geringer wird die Genauigkeit. Verschiedene Experimente lieferten eine maximale Geschwindigkeit von 60000 Steps/Sekunde, bei der der Motor noch akkurat war. Die maximale Beschleunigung wurde mit 1000000 Steps/Sekunde^2 ermittelt, bei der der Motor noch akkurat war. Bei höheren Werten ist der Stepper Motor nicht mehr in der Lage alle Schritte auszuführen, die ihm gesendet werden, sodass die Genauigkeit leidet.

\subsection{Third Set of Experiments: Optimizing Software}
- Durch Einsatz der FIFO-Queue nahm die Winkelgeschwindigkeit weniger extreme Werte an, siehe Abbildung \ref{fig:before_after_smoothing}. Die Werte vor der Glättung gehen betragsmäßig bis rund 700 rad/s, mit Ausreißern sogar bis 1400 rad/s. Nach der Glättung über die letzten 3 Winkel liegen die Werte bei maximal 400 rad/s.
\begin{figure}[htbp]
    \centering
    \includegraphics[width=0.4\textwidth]{img/before_smoothing.png}
    \includegraphics[width=0.4\textwidth]{img/after_smoothing.png}
    \caption{Comparison of the angle speed with and without the FIFO-Queue of Length 3}
    \label{fig:before_after_smoothing}
\end{figure}
- Der Einstz des Flags, welches 1 ist, wenn der Pole oben ist, um zu unterscheiden, ob der Winkel und Winkelgeschwindigkeit Null sind weil der Pole oben ist, oder weil der RL Agent die Nachrichten der Kamera schneller abruft, als diese neue Winkel bestimmen kann, hat funktioniert. Ohne diese Fehlerkorrektur tauchte dieses Problem in einem Datensatz bei 85 von 10240 Beobachtungen auf (0.83\%)

- Beim Vergleich von unterschiedlichen Action Spaces fällt auf, die Größe es Action Space keinen Einfluss auf die Performance unter unterschiedlichen Reward Funktionen hat. In Abbildung \ref{fig:action_space_comparison} sieht man den Mean Reward pro Episode über ein Window von 100 Episoden für die Reward Funktionen $r_{highVelocity}$, $r_{simple}$ und $r_{complex}$ mit $\alpha=5$, $\beta=1$, $\gamma=1$ und $\delta=0$ und Größen des Action Spaces von 101 Aktionen, 51 Aktionen und 2 Aktionen. Nach etwa 200000 Steps konvergiert die Reward Funktion. Da einfachere Modelle komplexeren Modellen in der Regel vorzuziehen sind, wird für die weiteren Experimente ein diskreter action space mit 2 Aktionen verwendet.
\begin{figure}[htbp]
    \centering
    \includegraphics[width=0.3\textwidth]{img/high_velocity_reward.png}
    \includegraphics[width=0.3\textwidth]{img/simple_reward.png}
    \includegraphics[width=0.3\textwidth]{img/complex_reward.png}
    \caption{Comparison of the convergence of the mean reward per episode for different reward functions and action spaces}
    \label{fig:action_space_comparison}
\end{figure}
- Beim Beobachten des RL Agent wurde auch schnell klar, dass ohne den Einsatz von time.sleep(0.1) der Rl Agent nicht in der Lage ist, das Pendel nach oben zu bringen, weil er zu viele Aktionen macht, die das cart nur schnell kurze distanzen hin und her fahren lässt. somit lässt sich kein Moment des Pendels aufbauen, welches aber nötig ist, um nach oben zu schwingen. Mit erhöhter Trainingszeit ist dies aber denkbar, dass der RL Agent dies selbst lernt. Die vielen schnellen Aktionen sind dann in der Balancierungsphase nötig

- Simple Reward Funktion führt nicht zum gewünschten Verhalten, die beobachteten Winkel sind oft sehr groß ($\pm\pi$, was der unteren Position des Pendels entspricht), siehe Abbildung \ref{fig:angle_simple_reward}. Der weiße Bereich bei $\pm$ 1 Radiant kommt daher, dass die Kamera keine Seite des Pendels sehen kann, wenn es sich in der Horizontalen befindet. Daher können in diesem Zustand auch keine Winkel gemessen werden. Die beobachtete Winkelgeschwidigkeit ist oft sehr groß, was darauf hindeutet, dass das Pendel sehr schnell schwingt. Damit ist es zwar möglich, das Pendel nach oben zu bringen, aber sehr schwer das Pendel oben zu balancieren (siehe \ref{fig:angle_velocity_simple_reward}). Der mittlere Reward pro Episode konvergiert zu rund -1600 (siehe \ref{fig:mean_reward_simple_reward}).
\begin{figure}[htbp]
    \centering
    \includegraphics[width=0.8\textwidth]{img/simple_reward_angle.png}
    \caption{Seen angles of the pendulum over 200000 steps with the simple reward function}
    \label{fig:angle_simple_reward}
\end{figure}
\begin{figure}[htbp]
    \centering
    \includegraphics[width=0.8\textwidth]{img/simple_reward_angular_velocity.png}
    \caption{Seen angular velocities of the pendulum over 200000 steps with the simple reward function}
    \label{fig:angle_velocity_simple_reward}
\end{figure}
\begin{figure}[htbp]
    \centering
    \includegraphics[width=0.8\textwidth]{img/simple_reward_mean_reward.png}
    \caption{Mean reward per episode with the simple reward function}
    \label{fig:mean_reward_simple_reward}
\end{figure}

- Reward Funktion von Kimura & Kobayashi (1999) hat auch keine gute Performance, zwar stabilisiert sich auch hier der Mean Reward (siehe \ref{fig:mean_reward_kimura1999}), aber die Winkel erreichen gar nicht mehr die 0 Radiant, es findet also kein Aufschwung statt (siehe \ref{fig:angle_kimura1999}). Die beobachtete Winkelgeschwidigkeit nimmt stark ab über die Zeit (siehe \ref{fig:angle_velocity_kimura1999}).
\begin{figure}[htbp]
    \centering
    \includegraphics[width=0.8\textwidth]{img/kimura1999_angle.png}
    \caption{Seen angles of the pendulum over 200000 steps with the reward function of Kimura & Kobayashi (1999)}
    \label{fig:angle_kimura1999}
\end{figure}
\begin{figure}[htbp]
    \centering
    \includegraphics[width=0.8\textwidth]{img/kimura1999_angular_velocity.png}
    \caption{Seen angular velocities of the pendulum over 200000 steps with the reward function of Kimura & Kobayashi (1999)}
    \label{fig:angle_velocity_kimura1999}
\end{figure}
\begin{figure}[htbp]
    \centering
    \includegraphics[width=0.8\textwidth]{img/kimura_mean_reward.png}
    \caption{Mean reward per episode with the reward function of Kimura & Kobayashi (1999)}
    \label{fig:mean_reward_kimura1999}
\end{figure}

- Reward Funktion, die von Escobar et al (2020) vorgeschlagen wurde, hat nicht zum Erfolg geführt, die beobachteten Winkel konvergieren auch nicht zu 0 Radiant, siehe Abbildung \ref{fig:angle_escobar2020}. Die beobachteten Winkelgeschwidigkeiten (\ref{fig:angle_velocity_escobar2020}) sind vergleichbar mit den beobachteten Winkelgeschwidigkeiten der einfachen Reward Funktion, der erziehlte Reward ist numerisch höher, aber abnehmend (\ref{fig:mean_reward_escobar2020}).
\begin{figure}[htbp]
    \centering
    \includegraphics[width=0.8\textwidth]{img/escobar2020_angle.png}
    \caption{Seen angles of the pendulum over 200000 steps with the reward function of Escobar et al (2020)}
    \label{fig:angle_escobar2020}
\end{figure}
\begin{figure}[htbp]
    \centering
    \includegraphics[width=0.8\textwidth]{img/escobar2020_angular_velocity.png}
    \caption{Seen angular velocities of the pendulum over 200000 steps with the reward function of Escobar et al (2020)}
    \label{fig:angle_velocity_escobar2020}
\end{figure}
\begin{figure}[htbp]
    \centering
    \includegraphics[width=0.8\textwidth]{img/escobar2020_mean_reward.png}
    \caption{Mean reward per episode with the reward function of Escobar et al (2020)}
    \label{fig:mean_reward_escobar2020}
\end{figure}

\section{Challenges and Limitations of Experiments}
- Da es sich nicht um eine Simulation, sondern um ein echtes System handelt, gibt es viele Herausforderungen, die in Simulationen nicht auftreten, wie z.B. die Erkennung der Winkel des Pendels in Echtzeit, verschiedene Delays, die durch die Hardware entstehen, und auch die akkurate Bestimmung der Winkelgeschwidigkeit: Winkeländerung geteilt durch sehr kleine Zeit ergibt einen großen Fehler, selbst bei kleinen Zeitabweichungen. Durch das Smoothing sollte der Effekt aber abgemildert werden.

- Um das Delay der Kamera zu bestimmen, wurde die aktuelle Zeit vor der Aufnahme des Bildes gespeichert und Vergleichen mit der Zeit, als die Berechnung des Winkel abgeschlossen war. Das Delay der Kamera und Bildverarbeitung schwankt um 0.1 Sekunden, das Delay aller Komponenten zusammen schwankt um 0.2 Sekunden, in Extremfällen beträgt es 0.7 Sekunden (siehe Abbildung \ref{fig:camera_delay_boxplot})
\begin{figure}
    \centering
    \includegraphics[width=0.8\textwidth]{img/total_delay_camera_delay_boxplot.png}
    \caption{Boxplot of the Camera Delay and Image Processing (labeled as Camera Delay) and the Total Delay of all Components (labeled as Total Delay)}
    \label{fig:camera_delay_boxplot}
\end{figure}
- Mit zunehmender Zeit scheint das insgesamte Delay größer zu werden, die Ausreißer nehmen zu (siehe \ref{fig:total_delay_over_time}). Ich kann mir nicht erklären warum und auch die Beobachtung, dass bei vielen Experimenten das Skript für die Bildverarbeitung nach etwa 1.3 Millionen Steps zwar nicht abgestürzt ist, aber keine Daten mehr geliefert hat, ist nicht erklärbar.
\begin{figure}
    \centering
    \includegraphics[width=0.8\textwidth]{img/total_delay_over_time.png}
    \caption{Delay of the Camera and Image Processing for each Observation}
    \label{fig:total_delay_over_time}
\end{figure}
%TODO: eventuell messen des Delays über IPC?
\chapter{Conclusion and Future Work}

\section{Summary of Findings}

\section{Potential Applications}

\section{Directions for Future Research}

%Anhang
\appendix
\phantomsection\addcontentsline{toc}{chapter}{Anhang}
\chapter{Appendices}

\section{Detailed Technical Diagrams or Source Code}

\section{Extended Data Analysis or Graphs}

% Festlegen des Bibliografie-Stils, in diesem Fall die deutsche Variante des Standard-Stils plain
% Für Bachelor- und Masterarbeiten sollte biblatex oder natbib verwendet werden. Eine kurze Suche bei Google führt auf die Paketdokumentationen,
% aus der sich alles Weitere ergibt.
% \bibliographystyle{gerplain}

% Einbinden der Bibliothek
% \bibliography{beispielbib}
% \thispagestyle{empty}

% Befehl zum Erstellen der Selbstständigkeitserklärung. Achtung: Es werden die Namen der Autoren, die in titlepage.tex festgelegt wurden, herangezogen und das Datum in \date{} verwendet.
% Befehl steht nur zur Verfügung, wenn das Paket VOSStatement eingebunden ist.
\cleardoublepage

\makestatement

% % % % % % % % % % % % % % % %
% Ende der document-Umgebung
% % % % % % % % % % % % % % % %
\end{document}